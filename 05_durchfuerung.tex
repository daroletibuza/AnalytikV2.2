\section{Durchführung}
\label{sec:durchfuerung}
\subsection{Versuchsteil 1: Aufnahme und Glättung des Polarogramms}
Nach Starten des Computersystems mit der Software \textsc{757 VA Computrace} werden die entsprechenden Voreinstellungen des Systems nach der Versuchsanleitung getroffen. Während dessen werden \SI{10}{\milli \liter} des Acetat-Puffer mit \SI{1}{\mol\per \liter} mit deionisierten Wasser auf \SI{100}{\milli \liter} zu \SI{0,1}{\mol\per \liter} aufgefüllt (vgl. mit Gl. \ref{gl:4}).
\begin{flalign}
\label{gl:4}
	n_{\text{vorher}}(\text{Acetatpuffer}) &= 	n_{\text{nachher}}(\text{Acetatpuffer})\\[2mm]
	c_{\text{vorher}}(\text{Acetatpuffer})*V_{\text{vorher}}(\text{Acetatpuffer}) &= c_{\text{nachher}}(\text{Acetatpuffer})*V_{\text{nachher}}(\text{Acetatpuffer})\\[2mm]
	c_{\text{nachher}}(\text{Acetatpuffer})&=\frac{V_{\text{vorher}}(\text{Acetatpuffer})}{V_{\text{nachher}}(\text{Acetatpuffer})}*c_{\text{vorher}}(\text{Acetatpuffer})\\[2mm]
	&=\frac{\SI{10}{\milli \liter}}{\SI{100}{\milli \liter}}*\SI{1}{\mol \per \liter}\\[2mm]
	&= \underline{\underline{\SI{0,1}{\mol \per \liter}}}											
\end{flalign} 

\newpage

In Folge der Verdünnung des Acetat-Puffers werden davon \SI{25}{\milli \liter} in das Messgefäß pipettiert, sowie weitere \SI{10}{\micro\liter} Triton-X, um die Oberflächenspannung der Lösung zu reduzieren. Die Messung durch das Messgerät wird mittels Mausklick gestartet und gibt ein Polarogramm aus.\\ 
Als nächster Schritt im Versuch wird versucht mittels fünfminütigen Spülens mit Stickstoff den gelösten Sauerstoff zu beseitigen. Ziel ist es dabei das zuvor ausgegebene Polarogramm zu glätten. \\
Zuletzt werden in diesem Versuchsteil \SI{1000}{\micro \liter} des Multielementcocktails mit je \SI{25}{\gram \per\liter} zugesetzt und ebenfalls ein Polarogramm davon aufgenommen.\\
Ein Vergleich und eine Auswertung der ausgegebenen Polarogramme erfolgt unter Abschnitt \ref{sec:ergebnisse}.

\subsection{Versuchsteil 2: Identifikation und Quantifizierung des Analyten Blei der Wasserprobe}
Die im Polarogramm gezeichneten Peaks entsprechen voraussichtlich den verschiedenen, enthaltenen Metall-Ionen im Messgefäß. Für die eindeutige Bestimmung des Peaks für Blei wird eine \SI{250}{\micro \liter} Blei-Standardlösung zugegeben. Die Peaks der restlichen Ionen bleiben gleich, während der Peak der für die Blei-Ionen steht, steigen wird. Die Peaks der jeweiligen Programme werden zum einen manuell und automatisch vom System bestimmt. Für nachfolgende Bestimmungen werden die Potentiale $E_{Start}$ und $E_{Ende}$ anhand des Peaks von Blei $\pm \SI{0,2}{\volt}$ festgelegt.\\
Die Konzentration des Analyten Blei wird im Folgenden mittels Standardadditionsmethode (beschrieben unter Abschnitt \ref{sec:theorie}) durch zweimaliges Aufstocken bestimmt.\linebreak
Dazu wird das Messgefäß gereinigt und der Maßkolben mit der Wasserprobe mit \SI{10}{\milli \liter} unverdünnter Pufferlösung $\left(\SI{1}{\mol\per\liter}\right)$ versetzt und mit destilliertem Wasser aufgefüllt. \SI{25}{\milli \liter} dieser Analysenprobe werden in das gereinigte Messgefäß mit \SI{10}{\micro\liter} Triton-X gegeben. Es werden nun die entsprechenden Parameter im Computerprogramm nach Praktikumsanleitung für zweimaliges Aufstocken mit \SI{200}{\micro\liter} Blei-Standard eingegeben. 
Die Messung startet ebenfalls per Mausklick und je nach Messfortschritt und Aufforderung des Programms wird der entsprechende Standard manuell zugegeben. Zum Schluss gibt das Programm die quantitativen, gemessenen Daten zusammengefasst in Graphen, Kalibriergeradengleichungen und Konzentration des Analyten aus.

\newpage
