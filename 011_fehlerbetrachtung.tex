\section{Fehlerbetrachtung}
\label{sec:fehler}

Fehler sind bei der Pipettierung der Lösungen aufgetreten. Zum ersten wurde abwechselnd von zwei Personen pipettiert und zum zweiten besitzen die verwendeten Vollpipetten Toleranzen. Die Toleranzen belaufen sich auf $\pm$\SI{0,03}{\milli\liter} für die \SI{10}{\milli\liter}-Pipette und auf $\pm$\SI{0,045}{\milli\liter} für die \SI{25}{\milli\liter}-Pipette. Die Lösungen waren nicht ständig agitiert, wodurch es zu absetzungsunterschieden und konzentrationsunterschieden innerhalb der Lösung gekommen sein könnte. Die Reinheit des Verwendeten Quecksilbers wikt sich ebenfalls auf das Messergebnis aus. Eine Spülzeit von \SI{30}{\second} wurde als ausreichend angenommen. Längere Spülzeiten könnten noch mehr Sauerstoff austreiben und den Störenden einfluss weiter verringern.

Die Konzentrationen der bereitgestellten Lösungen könnten vom erwarteten Wert abweichen. Das Alter der Lösungen ist unbekannt. Trotz vorherigem spülens

Auch das Messgerät \textsc{Metrohm 757 VA Computrace} unterliegt gewissen Toleranzen.

Ablesefehler aus dem Polarogramm sind in die erstellung der manuellen Kalibriergerade eingeflossen.