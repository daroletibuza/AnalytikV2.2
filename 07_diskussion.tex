\section{Diskussion}
\label{sec:diskussion}


\textbf{Interpretation mit Reaktionsgleichung Sauerstoff}\\
Die Reduktion des gelösten Sauerstoffs läuft ab einer Spannung von \SI{0}{\volt} wie in Gleichung (\ref{gl:o2red1}) und bei circa \SI{1}{\volt} entsprechend der Gleichung (\ref{gl:o2red2}) ab.
\begin{flalign}\label{gl:o2red1}
\ce{O2 + 2 e^- + 2 H2O -> H2O2 + 2 OH-}
\end{flalign}
\begin{flalign}\label{gl:o2red2}
\ce{H2O2 + 2 e^- -> 2 OH^-}
\end{flalign}
\vspace*{3mm}\\
\textbf{Vergleich Polarogramm geglättet und ungeglättet}\\
Das ungeglättete Polarogramm (vgl. Abb.\ref{fig:daten_farbig} in gelb) zeigt den Stomstärkeverlauf in Anwesenheit von Sauerstoff. Es treten Peaks auf, wenn die Reduktionsreaktionen (\ref{gl:o2red1}) bei etwa \SI{0}{\volt} und (\ref{gl:o2red2}) bei etwa \SI{-1}{\volt} ablaufen. Nach erfolgter Spülung mit Stickstoff ist der Sauerstoff aus der Lösung verschwunden. Er kann nicht mehr reduziert werden und stört darum auch nicht mehr das Polarogramm (vgl. Abb.\ref{fig:daten_farbig} in schwarz) welches damit als geglättet anzusehen ist. Es besitzt nun einen quasi horizontalen Verlauf.\\
\vspace*{3mm}\\
\textbf{Zuordnung Begründen}\\
Die Peaks können den enthaltenen Elementen aufgrund ihrer Stellung in der elektrochemischen Spannungsreihe zugeordnet werden. Je edler ein Metall ist, umso höher ist auch das entsprechende Standardelektrodenkpotential und umso eher wird es reduziert. \\
oder zweite Peak kann besonders sicher dem Blei zugeordnet werden, da durch wissentliche Erhöhung der Bleikonzentration ein größerer Ausschlag erkennbar ist.\\
\vspace*{3mm}\\
\textbf{Vergleich mit Grenzwert für Trinkwasser}\\
Der Grenzwert für Trinkwasser liegt laut der aktuell gültigen Fassung der Trinkwasserverordnung \cite{TWV} bei \SI{10}{\micro\gram\per\liter}. Den statistischen Grenzwerttests im Folgenden ist zu entnehmen, dass selbiger eingehalten wird.
\newpage
\subsection*{Statistischer Grenzwerttest zur Unterschreitung des Trinkwassergrenzwertes für Blei}
Gegebene Parameter (vgl. Tab.\ref{tab:mauell,automatisch})
\begin{itemize}
	\item Mittelwert $\bar{x}$=  \SI{6,591}{\milli\gram\per\liter}
	\item Grenzwert $x_{Grenz}$= \SI{10}{\milli\gram\per\liter}
	\item Standardabweichung s= \SI{0,113}{\milli\gram\per\liter}
	\item N=3
\end{itemize}
\subsubsection*{Grenzwerttest für eine Sicherheit von 95\%}
\begin{flalign}
t_{EMP;95}&=\frac{\bar{x}-x_{Grenz}}{s}*\sqrt{N}\\
&=\frac{\SI{6,591}{\milli\gram\per\liter}-\SI{10}{\milli\gram\per\liter}}{\SI{0,113}{\milli\gram\per\liter}}*\sqrt{3}\\
&= -52,2528
\end{flalign}
Der tabellierte Wert für $t_{CRIT}$ für einen einseitigen Test mit einer Sicherheit von 95\% bei einem Freiheitsgrad von 2 lautet 2,920.

$$t_{EMP}<-t_{CRIT} $$
$$-52,2528< - 2,920$$

Damit ist bewiesen, dass der Grenzwert mit einer Wahrscheinlichkeit von 95\% eingehalten ist.

\subsubsection*{Grenzwerttest für eine Sicherheit von 99,9\%}
\begin{flalign}
t_{EMP;95}&=\frac{\bar{x}-x_{Grenz}}{s}*\sqrt{N}\\
&=\frac{\SI{6,591}{\milli\gram\per\liter}-\SI{10}{\milli\gram\per\liter}}{\SI{0,113}{\milli\gram\per\liter}}*\sqrt{3}\\
&= -52,2528
\end{flalign}
Der tabellierte Wert für $t_{CRIT}$ für einen einseitigen Test mit einer Sicherheit von 99,9\% bei einem Freiheitsgrad von 2 lautet 22,327.

$$t_{EMP}<-t_{CRIT} $$
$$-52,2528< - 22,327$$

Damit ist bewiesen, dass der Grenzwert mit einer Wahrscheinlichkeit von 99,9\% eingehalten ist.


